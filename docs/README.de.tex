
\documentclass{scrartcl}

\usepackage{graphicx}
\usepackage[x11names, dvipsnames*, svgnames]{xcolor}
\setlength{\fboxsep}{.05\linewidth}
\usepackage[hyperindex=false, pdfpagelabels,
pageanchor, hyperfootnotes=false, bookmarksopen,
pdfpagemode=UseOutlines]{hyperref}

\usepackage{mathspec}
\usepackage{fontspec}
\usepackage{xunicode}
\usepackage[no-sscript]{xltxtra}

\setmainfont[Mapping=tex-text,  Scale=1.03, ItalicFeatures={SmallCapsFont=AlegreyaSC-Italic}]{Alegreya}
\setsansfont[Mapping=tex-text, Scale=1.03]{Alegreya Sans}

\usepackage{polyglossia}
\setdefaultlanguage{english}

\usepackage{microtype}
\usepackage{xspace}

\setlength{\parindent}{0pt}
\sloppy


\title{Einführung in LaFiC}
\author{Sebastian Meisel}

\begin{document}

\maketitle


{LaFiC bedeutet \textit{layout and format in comments}, also „Layout
und Format in Kommentaren“, denn sämtliche Formatierungen
werden in LAFIC in Kommentarzeilen ausgeführt. Das
ermöglicht eine \emph{vollständige} Trennung von Inhalt und
Gestalt.\\}

\section{Warum LaFiC}

{Ich arbeite nun schon viele Jahre mit \LaTeX\  / \XeLaTeX\ . Ich
schreibe vor allem Prosa (ganz frei von mathematischen
Formeln). Dabei fand ich es zum Teil störend, dass ich immer
erst mit der Preamble beginnen muss und nicht einfach
losschreiben kann.\\}

{Daraufhin probierte ich markdown/multimarkdown aus. Dabei
störte mich aber, dass es sehr unflexible ist. Außerdem ist
die Syntax an manchen Stellen etwas kryptisch, was mich
störte. Außerdem funktionieren die erstellten \LaTeX\ -Dateien
zwar irgendwie, sind aber auch sehr kryptisch aufgebaut.\\}

{Außerdem hatte ich immer die Worte meines Vaters im Ohr, der
sagte, er wolle am liebsten, wie einst mit der
Schreibmaschiene losschreiben können und dennoch ein
ordentliches Ergebnis bekommen.\\}

{Als letzte Motivation kam hinzu, dass ich immer wieder
darüber nachdachte, wie sich Inhalts- und
Gestaltungselemente des Textes noch sauberer trennen lassen,
als dies mit bisherigen Ansätzen möglich ist.\\}

{Mit LaFiC kann ich direkt losschreiben und bekomme von
anfang an ein ordentlich strukturiertes Ergebnis als Html-
und (Xe)LaTeX-\footnote{Die Standard-Vorlagen, die LaFiC nutzt basieren auf \XeLaTeX\ , da ich immer in Utf-8 schreibe. Die Nutzung von \LaTeX\  sollte aber auch möglich sein.}\xspace Datei, wodurch auch eine Ausgabe als Pdf
möglich ist.\\}

{Die eigentliche Formatierung geschieht erst im
Anschluss. Als mein eigener Lektor gehe ich durch den Text
und formatiere ihn mit für Menschen lesbaren und
verständlichen Kommentaren.\\}

\part{Installation und Benutzung}

\section{Vorraussetzungen}

{LaFiC setzt Perl > 5.10.1 vorraus (getestet unter Perl 5.26.1).\\}

{Das Standard \LaTeX\ -Template setzt eine aktuelle
\XeLaTeX\ -Installation sowie graphix, hyperref, microtype und
xspace voraus.\\}

{Die Emacs-Einbindung wurde unter Gnu Emacs 25.2.2 getestet.\\}

{\texttt{lafic2pdf} setzt zudem latexmk (getestet mit Version 4.41) vorraus.\\}

\section{Installation}
\label{Installation}

{Holen Sie den Quelltext von \href{https://github.com}{github}:\\}

\begin{verbatim}
git clone https://github.com/SebastianMeisel/lafic.git
\end{verbatim}


{Fügen Sie den Pfad zu \$PATH Umgebungsvariable hinzu:\\}

\begin{verbatim}
export PATH=${PATH}:~/lafic
\end{verbatim}


{Zur Nutzung des Major Modes in Emacs, kopieren (oder
verlinken) die Dateien \texttt{lafic-mode.el} und
\texttt{lafic-german-keywords.el} in ein Verzeichnis in Ihrem
\texttt{load-path}. Fügen Sie folgende Zeilen in Ihre Init-Datei
(\texttt{\textasciitilde /.emacs}) ein:\\}

\begin{verbatim}
(setq lafic-use-german t)
(require 'lafic-mode)
\end{verbatim}


{Wenn Sie keine Unterstützung von deutschen Schlüsselwörtern
wünschen oder benötigen, lassen Sie die erste Zeile weg.\\}

\section{Nutzung}

{Momentan besteht die LaFiC-Distribution aus drei
Skripten. Der Aufruf erfolgt jeweils analog:\\}

\begin{verbatim}
# lafic2html Datei.lafic
# lafic2tex Datei.lafic
# lafic2pdf Datei.lafic
\end{verbatim}


{Letzteres ruft zuerst \texttt{lafic2tex} und \texttt{latexmk} auf.\\}

{Durch den Aufruf dieser drei Skripte würden folgende Dateien
erstellt:\\}

\begin{verbatim}
Datei.html
Datei.tex
Datei.pdf
\end{verbatim}


\section{Nutzung des LaFiC-Major-Modes in Emacs}

{Nachdem Sie \texttt{lafic-mode.el} wie oben (siehe \nameref{Installation}\xspace )
beschrieben in Emacs eingebunden haben, wird der
LaFiC-Major-Mode beim Öffnen einer Datei mit der Endung
\texttt{*.lafic} aktiviert.\\}

{Dadurch stehen eine einfache Syntaxhervorhebung und einige
Tastenkürzel zur Verfügung. Als Prefix wird dabei jewweil
\texttt{C-c} verwendet. Die Tastenkürzel orientieren sich an AUCTeX,
sodass man gut zurecht kommen sollte, wenn man damit schon
gearbeitet hat.\\}

\part{Schreiben mit LaFiC}

\section{Zeilen und Absätze}

{Der Inhalt wird durch die Unterscheidung von \emph{Zeilen} und
\emph{Absätzen} gegliedert.\\}

{Dabei besteht der Unterschied nicht so sehr in der
Länge. Vielmehr unterscheiden sich Zeilen von Absätze
dadurch, dass sie kein Satzschlusszeichen (., ?, !, :).
Wenn nicht anders festgelegt, werden sie als Überschriften
interpretiert.\\}

{Die erste \emph{Zeile} wird als Titel interpretiert und zu \textbackslash title
umgewandelt, wenn die Datei in \LaTeX\  umgewandelt wird, bei
Html-Ausgabe wird es in <h1> umgewandelt.\\}

{Weitere \emph{Zeilen} werden in <h3> (\textsc{Html}) oder \textbackslash section
(LaTeX) umgewandelt, wenn es nicht anders angegeben wird.\\}

{Auf diese Weise können einfache Texte ganz ohne Formatierung
gegliedert werden.\\}

\section{Kommentare}

{Man kann im Text jeder Zeit Kommentare einführen in dem man einen Absatz einfügt, der mit zwei \%-Zeichen beginnt:\\}

\begin{verbatim}
  %% Dies ist ein Kommentar.

  %% Dies ist ein längerer Kommentar. Es ist wichtig, dass
  %% Kommentare immer durch eine Leerzeile vom eigentlichen
  %% Inhalt getrennt sind.
\end{verbatim}




\section{Formatierte Absätze}

{Absätze können formatiert werden, in dem eine Zeile
vorangestellt wird, die mit einem \%-Zeichen beginnt dem ein
Schlüsselwort folgt. Ist das Schlüsselwort bekannt, wird
die entsprechende Umgebung (LaTeX), bzw. der entsprechende
Block (Html) ausgegeben. Ansonsten dient das Schlüsselwort
(umgewandelt in Kleinschreibung) als Name der Umgebung
(LaTeX), bzw. eines <div>-Blocks (Html).\\}

\begin{verbatim}
  % Zentriert
  Dieser Absatz ist zentriert.
\end{verbatim}


\begin{center}
Dieser Absatz ist zentriert.
\end{center}


{Folgen zwei Absätze mit dem selben Schlüsselwort
hintereinander, werden sie in einer Umgebung / einem Block
zusammengefasst.\\}

\begin{verbatim}
  % Zitat
  Dies ist ein Zitat.

  % Zitat
  Hier geht das Zitat weiter.
\end{verbatim}


{Wird zu:\\}

\colorbox{gray!75}{\parbox{\linewidth}{%%
\begin{quote}
Dies ist ein Zitat.

Hier geht das Zitat weiter.
\end{quote}

}
}

{Zur Zeit werden folgende Schlüsselworte unterstützt:\\}

\begin{itemize}
\item Zitat für quote-Umgebung / <blockquote>-Block.
\item Langzitat / LZitat für quotation-Umgebung / <blockquote>-Block.
\item Zentriert / Z für center-Umgebung / <center>-Block
\end{itemize}


\section{Formatierte Zeilen}

{Zeilen werden auf die gleiche Weise formatiert, nur werden sie in Makros (LaTeX) oder <span>-Ids (Html) umgewandelt. Zeilen sind in der Regel Überschriften.\\}

\begin{verbatim}
    % Unterabschnitt
    Das ist ein Unterabschnitt
\end{verbatim}


\subsection{Dies ist ein Unterabschnitt}

{Folgende Schlüsselwörter werden zur Zeit unterstützt:\\}

\begin{itemize}
\item Überschrift für addsec / H3
\item Titel/ Überschrift1  für title / H1
\item Teil / Überschrift2 für part / H2
\item Kapitel für chapter / H2\footnote{Wird nicht von der Standard-Vorlage unterstützt. Es muss eine \LaTeX\ -Klasse verwendet werden, die \textbackslash chapter unterstützt.}\xspace  
\item Abschnitt / Überschrift3 für section / H3
\item Unterabschnitt / Überschrift4 für subsection / H4
\item Unterunterabschnitt / Überschrift5 für subsubsection / H5
\item Absatz / Überschrift6 für paragraph / H6
\end{itemize}


\section{Textformatierung im Absatz}

{Wörter und Textabschnitte in einem Absatz werden in einem
Block nach dem Absatz formatiert. Jede Zeile beginnt mit
einem »\texttt{\%}« gefolgt von:\\}

\begin{itemize}
\item dem Wort
\item dem ersten und letzten Wort den Abschnittes,
  getrennt durch3~Punkte, bzw. durch »…«.
\end{itemize}


{If the keyword is
unknown, it is converted to a macro (LaTeX) oder <span>
(Html) name.\\}

\section{Images}

{The simplest way to put an image into a lafic file is a
line with the image name, with a know extention: png, jpg,
jpeg, gif.\\}

{\centering\includegraphics[width=.50\linewidth]{Image.png}}

{Note that this will not put an figure environment in \LaTeX\ 
files, so the image won't float this way. For this to
achieve to have to put \%image, \%img or \%figure before the
line. You don't need the extention then.\\}

\begin{figure}[hbt]
\raggedleft%%
\includegraphics[height=.40\textheight]{Image.png}
\caption{Mars and Moon}
\label{bild1}
\end{figure}


\section{Farben}

\colorbox{red}{\parbox{\linewidth}{%%
{\raggedleft%%
\textcolor{white}{%%
Die Hintergrundfarbe eines Absatzes kann über den Parameter
\textcolor{green}{Hintergrund} gesetzt werden; die Textfarbe über \textcolor{green}{Farbe}. Um in
einem Absatz einzelne Wörter oder einen Bereich farbig zu
gestalten, können analog die Schlüsselwörter \textcolor{green}{Farbe} und
\textcolor{green}{Hintergrund} verwendet werden.}\\}
}
}

\begin{verbatim}
  Die Hintergrundfarbe eines Absatzes kann über den Parameter
  Hintergrund gesetzt werden; die Textfarbe über Farbe. Um in
  einem Absatz einzelne Wörter oder einen Bereich farbig zu
  gestalten, können analog die Schlüsselwörter Farbe und
  Hintergrund verwendet werden.
  % Hintergrund = rot
  % Ausrichtung = right
  % Farbe = weiß
  % Hintergrund( ): Farbe: grün
  % Farbe: Farbe: grün
\end{verbatim}



\end{document}

%%% Local Variables:
%%% mode: latex
%%% TeX-master: t
%%% End:
