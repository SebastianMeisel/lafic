
\documentclass{article}

\usepackage{graphicx}
\usepackage{xcolor}
\definecolor{gray}{gray}{.75}
\setlength{\fboxsep}{.05\linewidth}
\usepackage[hyperindex=false, pdfpagelabels,
pageanchor, hyperfootnotes=false, bookmarksopen,
pdfpagemode=UseOutlines]{hyperref}

\usepackage{mathspec}
\usepackage{fontspec}
\usepackage{xunicode}
\usepackage[no-sscript]{xltxtra}
\usepackage{polyglossia}
\setdefaultlanguage{english}

\usepackage{microtype}
\usepackage{xspace}

\setlength{\parindent}{0pt}
\sloppy


\title{Introduction to LaFiC}
\author{Sebastian Meisel}

\begin{document}

\maketitle


{LaFiC means \textit{layout and format in comments}, as all layout and
format information is put into comment lines. So layout and
content are \emph{fully} separated. For details see \nameref{Writing}\xspace .\\}

\part{Installation}
\label{}

{Get source from \href{https://github.com}{github} using:\\}

\begin{verbatim}
git clone https://github.com/SebastianMeisel/lafic.git
\end{verbatim}


{Add lafic directory to \$PATH, e.g.:\\}

\begin{verbatim}
export PATH=${PATH}:~/lafic
\end{verbatim}


{See \texttt{lafic-mode.el} for installation instructions, if you want
to use in in Gnu Emacs\footnote{GNU Emacs ist als freie Software unter der GNU General Public License erhältlich und läuft auf den meisten heute üblichen Betriebssystemen (Unix, GNU/Linux, macOS und Windows).}\xspace .\\}

\part{Writing text in LaFiC}
\label{Writing}

\section{Lines and paragraphs}
\label{}

{The content is presented in two forms, which also include
the most basic layout: There are \emph{lines} and \emph{paragraphs}.\\}

{The difference is not the length, but lines include
none of the punctation marks ».«, »?«, »!«, \emph{»:«}. If no
further layout information is provided, these are
interpreted as headings.\\}

{The \emph{first} line is interpreted as the title and presented as
 <h1>, when converted to Html, and \textbackslash title, when converted to
 \LaTeX\ .\\}

{Further lines will be converted to <h3> (Html) or \textbackslash section
(LaTeX), if no otherwise specified.\\}

{This way simple Documents Html may be structured with no explicit
layout information at all.\\}

\section{Formation lines}
\label{}

{You can add layout in\emph{formations and formations in lines}
\emph{before} and \emph{after} the paragraph. These \emph{format lines} start
with a \%-char (with no leading spaces) and must not be
separated frome the paragraph by blank lines.\\}

\subsection{Paragraph and line formats}
\label{}

{You can specify a format for a paragraph or line by a
leading format line with a single \emph{keyword}. The \emph{keyword} will
be translated to a block element name (Html) or a macro /
environment (LaTeX). The use of \emph{keyword}s is unique to lafic,
as you can define multiple \emph{keyword}s for on block element,
macro or environment. So \emph{h4}, \emph{heading4} and \emph{subsection} will all be
translated to <\emph{h4}> (Html) and \textbackslash \emph{subsection} (LaTeX).\\}

{You can force lines (with no ., ?, !, :) to be interpreted
as paragraphs with a leading paragraph format (like
quotation) or a leading empty format line – just the \%-char.\\}

\begin{verbatim}
  % quotation
  This is a quotation

  %
  This is a paragraph
\end{verbatim}


\colorbox{gray!75}{\parbox{\linewidth}{%%
\begin{quotation}
This is a quotation
\end{quotation}

}
}

\colorbox{gray!75}{\parbox{\linewidth}{%%
{
This is a paragraph\\}
}
}

{On the other hand you can…\\}

\section{Comments}
\label{}

{You can add comments to your text, by starting a single line
or each \emph{line} of a paragraph with a \% char with no leading
spaces. These lines or paragraphs must, however, be
separated by empty line from the content.\\}

\begin{verbatim}
  % This is a comment.

  % This is a longer comment, that spreads over several
  % lines. It is important that it is not connected to a line
  % of the general content.
\end{verbatim}


{“It is recommended, however, to start comments with two \% chars.
Else there may occur ‘ problems’, when there” is a »:« somewhere
in the comment. You also can start a longer comment this way
and don't need to repeat it ‘ in every’ line.\\}

\begin{verbatim}
  %% This is a comment! No mistake!
  Even when you go on with no leading % it's still a comment.
\end{verbatim}



\section{Formated paragraphs}
\label{}

{Paragraphs can be formated by adding a line before the
paragraph, that starts with a \% char, followed by a single
word. There are some predefined keywords, like quote or
quotation for – well a quotation.\\}

\begin{verbatim}
  % quote
  This is a quotation.
\end{verbatim}


\begin{quote}
This is a quotation.
\end{quote}


{If the keyword is unknown, it will be converted to an environment
name in \LaTeX\  or the name of a <div> in Html.\\}

\section{Formated lines}
\label{}

{Line are formated in the same way, only they are converted
to macros (LaTeX) oder <span> names (HTML). Know keywords
are H1~… H6 for headings.\\}

\begin{verbatim}
  % heading4
  This is a subsection
\end{verbatim}


\subsection{This is a subsection}
\label{}

\section{Inline formation}
\label{}

{If you want to format words or sequences in a paragraph (or
line if needed), you add format lines with a leading \% after
a paragraph. It has two parts:\\}

\begin{enumerate}
\item the word or the sequence to be formated in the form
  start…end. 
\item a keyword.
\end{enumerate}


{The both are separated by a »:«.\\}

{Known environment keywords are e.g. quote or quotation.\\}

{If the keyword is unknown, it is converted to a macro
(LaTeX) oder <span> (HTML) name.\\}

\section{Lists}
\label{}

{Lists are the only things, that need some kind of
markup. You have to start each topic of the list with one of
the following chars: –, *, +, -. It doesn't matter, which one you
choose. You may indent the lines, but that has no influence
on the layout.\\}

\begin{verbatim}

* Top 1.
- Top 2.
\end{verbatim}


\begin{itemize}
\item Top 1.
\item Top 2.
\end{itemize}


{For multilevel lists, you have to choises, to raise or
decrease the level: The clean LaFiC style would be,
to start a new paragraph and add the keyword \texttt{»\% level+«}
or \texttt{»\% level-«} at the end.\\}

\begin{verbatim}

  * Top 1.
  * Top 2.


  * Top 2a.
  * Top 2b.
  % level+
\end{verbatim}


\begin{itemize}
\item Top 1.
\item Top 2.

\begin{itemize}
\item Top 2a.
\item Top 2b.
\end{itemize}

\end{itemize}


{Or you can write the list in one paragraph, marking the
raise or decrease of the level with a > or < at the
beginning of a single line.\\}

\begin{verbatim}

  * Top 1.
  * Top 2.
  >
    * Top 2a.
    * Top 2b.
  <
  * Top 3
\end{verbatim}


\begin{itemize}
\item Top 1.
\item Top 2.
\begin{itemize}
\item Top 2a.
\item Top 2b.
\end{itemize}
\item Top 3
\end{itemize}


\section{Images}
\label{}

{The simplest way to put an image into a LaFiC file is a
line with the image name, with a know extention: png, jpg,
jpeg, gif.\\}

\begin{verbatim}
  Image.png
  % height = 40%
\end{verbatim}


{\includegraphics[height=.40\textheight]{Image.png}}

{Note that this will not put an figure environment in \LaTeX\ 
files, so the image won't float this way. For this to
achieve to have to put \% image, \%img or \%figure before the
line. You don't need the extention then.\\}

\begin{verbatim}
  %image
  Image.png
  % width = 40%
  % caption = "Moon and Mars"
\end{verbatim}


\begin{figure}[hbt]
%%
\includegraphics[width=.40\linewidth]{Image.png}
\end{figure}
\end{document}

%%% Local Variables:
%%% mode: latex
%%% TeX-master: t
%%% End:
