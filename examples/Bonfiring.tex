
\documentclass{scrartcl}

\usepackage{graphicx}
\usepackage[x11names, dvipsnames*, svgnames]{xcolor}
\setlength{\fboxsep}{.05\linewidth}
\usepackage[hyperindex=false, pdfpagelabels,
pageanchor, hyperfootnotes=false, bookmarksopen,
pdfpagemode=UseOutlines]{hyperref}

\usepackage{mathspec}
\usepackage{fontspec}
\usepackage{xunicode}
\usepackage[no-sscript]{xltxtra}

\setmainfont[Mapping=tex-text,  Scale=1.03, ItalicFeatures={SmallCapsFont=AlegreyaSC-Italic}]{Alegreya}
\setsansfont[Mapping=tex-text, Scale=1.03]{Alegreya Sans}

\usepackage{polyglossia}
\setdefaultlanguage{english}

\usepackage{microtype}
\usepackage{xspace}

\setlength{\parindent}{0pt}
\sloppy

%%%%%%%%%%%%%%%%%%%%%%%%%%%%%%%%%%%%%%%%
%% Custom macros and environments
%%%%%%%%%%%%%%%%%%%%%%%%%%%%%%%%%%%%%%%%
\newenvironment{frage}{\itshape}{}
\newenvironment{antworta}{
  \begin{quotation}
  }{
  \end{quotation}
}
\newenvironment{antwortb}{
  \begin{quotation}
  }{
  \end{quotation}
}

\newcommand{\name}[1]{\textbf{#1}}

\usepackage{tcolorbox}

\newenvironment{infobox}{%%
  \begin{tcolorbox}[width=\linewidth,
     boxsep=10pt,
     left=0pt,
     right=0pt,
     top=10pt,
     title=Zu den Personen,
     ]%%
   }{%%
   \end{tcolorbox}%%
}

\title{Vereinsamt unsere Jugend\\
\emph{Kontroverses Interview zum Bonfiring} \\}
\author{Sebastian Meisel}

\begin{document}

\maketitle


{Der wachsende Trend zum sogenannten \emph{Bonfiring}, löst bei
vielen Älteren in der Gesellschaft Sorgen aus. Wir wollen
mit drei Experten zu diesem Thema ergründen, inwieweit diese
Sorgen begründet, oder ob sie möglicherweise übertrieben sind.\\}

\begin{frage}
Herr~Müller, Bonfiring ist ein ziemlich neues
Phänomen. Könnten Sie kurz zusammenfassen wie genau das
funktioniert?
\end{frage}


\begin{antworta}
\name{Müller}: Nun, Menschen treffen sich an Orten mit schwacher
\emph{Linked Mind} Abdeckung. Das sind ausschließlich
abgelegene Gebiete, sodass man dort in der Regel campt
und sich am Lagerfeuer zusammensetzt. Daher der Begriff
„Bonfiring“, vom englischen „bonfire“ für Lagerfeuer.

Dort setzten sie ein \emph{Scammer} genanntes Gerät ein, um sich
für eine begrenzte Zeit ganz vom Netz zu trennen. Sie sich
dann über Themen aus, ohne dass dies von außen überwacht
werden kann.
\end{antworta}


\begin{antwortb}
\name{Scott}: Und ohne, dass die implementierten Schutzmechanismen
des Linked Mind wirken, mit denen wir seit Jahrzehnten
unsere Jugend vor negativen Einflüssen schützen.
\end{antwortb}


\begin{antworta}
\name{Cobina}: Sie meinen: „vor dem Leben, vor Erfahrungen,
vor dem Erwachsenwerden“.
\end{antworta}


\begin{antwortb}
\name{Scott}: Ich meine vor Einflüssen, die nachweislich einen
negativen Einfluss auf die soziale Prognose
Heranwachsender haben – wie ich in meiner Studie
nachgewiesen habe und wie es von einer Unzahl anderer –
wenn auch kleinerer Studien bestätigt wurde. Es wundert
mich im Übrigen, dass ausgerechnet Sie, ihre eigene
Erfindung nicht zu würdigen wissen~…
\end{antwortb}


\begin{antworta}
\name{Cobina}: An der Implementierung dieser
„Schutzmechanismen“ war ich nicht beteiligt, aber ich
glaube durchaus, dass sie ihre Berechtigung haben. Ich
glaube aber auch, dass es inzwischen an Freiräumen für
unsere Jugend fehlt. Zu einer gesunden Entwicklung
gehören die eben auch. Wir ziehen Kinder groß, statt sie
erwachsen werden zu~…
\end{antworta}


\begin{frage}
Hier muss ich kurz dazwischen gehen. Wir sind schon mitten
  in der Diskussion und das ist einerseits gut so, aber wir
  wollen unsere Leser doch auch mitnehmen. Deswegen bitte
  ich nun zunächst Sie Herr~Scott: Sie sehen das Bonfiring
  als Gefahr für unsere Jugend an – könnten Sie bitte kurz
  die wesentlichen Gefahren aus Ihrer Sicht darstellen?
\end{frage}


\begin{antworta}
Scott: Das tue ich sehr gerne. Wir alle sind uns sicher
    einig, dass das Linked Mind die größte Errungenschaft
    der Menschheit ist. Es verbindet uns zu einer Einheit
    und schenkt zugleich jedem Menschen die Freiheit, so zu
    leben, wie es gut für sie oder ihn ist, weil wir
    einander nun ganz anders verstehen können. Wenn junge
    Menschen sich nun aus dem Linked Mind ausklinken, so
    birgt das viele Gefahren – auch deshalb, weil sie sich
    so zugleich aus dem Schutzbereich begeben, den das
    Linked Mind darstellt. Sie können so mit Ideen
    konfrontiert werden, die ihnen und ihrer sozialen
    Entwicklung schaden.
\end{antworta}


\begin{infobox}
\name{Amma Cobina} ist 118~Jahre alt. Sie lebt als Ashanti im
Gebiet des Freien Afrika. Bis zu ihrem 73-sten Lebensjahr
arbeitete sie als IT-Spezialistin im Bereich
Neuro-Interface-Forschung und gilt als eine der Mütter des
Linked Mind.

\name{George Scott} ist 98~Jahr alt und forscht und lehrt im
Bereich Soziologie in Oxford, GB. Er leitete die
umfassendste internationale Langzeitstudie zu den positiven
Auswirkungen des Linked Mind auf die Gesellschaft.

\name{Harald Müller} ist ein deutscher Aktivist der den
Widerstand in den deutschen Gebieten der Abendländischen
Allianz stärkt. Dafür setzt er unter anderem auf Bonfiring.
\end{infobox}
\end{document}

%%% Local Variables:
%%% mode: latex
%%% TeX-master: t
%%% End:
